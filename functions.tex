\catcode`@=11

\let\@xp=\expandafter

\def\@nul{\noexpand\@nul}
\def\@xlet#1#2{\@xp\let\@xp#1#2}
\def\@afterfi#1\fi{\fi#1}
\def\@tofi#1#2\fi{\fi#1}
\def\@gobble#1{}

\def\literal{\afterassignment\@tpp@temp \def\@tpp@temp}
\def\m@strip{\@xp\@gobble\string}

\def\@prepeat#1#2#3{%
    \unless\ifnum#1>#2 %
        \@afterfi#3{#1}%
        \@xp\@prepeat\@xp{\the\numexpr #1+1\relax}{#2}{#3}%
    \fi%
}

\def\prepeat#1#2{\@prepeat{1}{#1}{#2}}

\def\@repeatit#1#2#3{%
    \unless\ifnum#1>#2 %
        \@afterfi#3%
        \@xp\@repeatit\@xp{\the\numexpr #1+1\relax}{#2}{#3}%
    \fi%
}

\def\repeatit#1#2{\@repeatit{1}{#1}{#2}}

% Construct an empty primitive array
% @param #1:    The name of the array
\def\p@array#1{\@xp\def\csname p@array@#1\endcsname{}}

\def\@p@array@init#1#2,{%
    \def\@p@init@temp{#2}%
    \ifx\empty\@p@init@temp %
    \else%
        \@afterfi\p@array@pushback{#1}{#2}%
        \@p@array@init{#1}%
    \fi% 
}

% Construct and initialize a primitive array
% @param #1:    The name of the array
% @param #2:    The values to initialize the array with
\def\p@array@init#1#2{%
    \p@array{#1}%
    \@p@array@init{#1}#2,,%
}

% Push an element to the end of a primitive array
% @param #1:    The name of the array
% @param #2:    The element to push back
\def\p@array@pushback#1#2{%
    \@xlet{\p@array@temp}{\csname p@array@#1\endcsname}%
    \def\@temp@def{\@xp\def\csname p@array@#1\endcsname}%
    \@xp\@temp@def\@xp{\p@array@temp\p@exe{#2}}%
}

% Get the length of a primitive array
% @param #1:    The name kd the array
\def\p@array@len#1{{%
    \count@=0 %
    \def\p@exe{\advance\count@ by 1}%
    \csname p@array@#1\endcsname%
    \xdef\tpp@return{\the\count@}%
}}

% Get an element at an index in a primitive array
% @param #1:    The name of the array
% @param #2:    The index (starting from 0) to get
% @return:      The element is placed globally in the \tpp@return register.
%               If the index is out of bounds, the register is let equal to \relax.
\def\p@array@index#1#2{{%
    \let\tpp@register=\relax%
    \count@=0 %
    \def\p@exe##1{%
        \ifnum\count@=#2 %
            \gdef\tpp@return{##1}%
        \fi%
        \advance\count@ by 1 %
    }%
    \csname p@array@#1\endcsname%
}}

\def\p@array@foreach#1#2{%
    \def\p@exe##1{#2{##1}}%
    \csname p@array@#1\endcsname%
}

\def\@create@param@list#1{\prepeat{#1}{\literal##1{#####1}}}

\def\@scope@stack@count{0}
\def\@scope@stack@def@count{0}

\def\@scope@stack@push#1{%
    \@xp\edef\csname @scope@stack@element@\@scope@stack@count\endcsname{\m@strip#1}%
    \@xp\let\csname @scope@stack@def@\@scope@stack@count\endcsname=#1%
    \edef\@scope@stack@count{\the\numexpr \@scope@stack@count + 1\relax}%
    \edef\@scope@stack@def@count{\the\numexpr \@scope@stack@def@count + 1\relax}%
}

\def\@scope@stack@pop{%
    \edef\@scope@stack@count{\the\numexpr \@scope@stack@count - 1\relax}%
    \def\@tpp@temp{\@xp\let\csname \csname @scope@stack@element@\@scope@stack@count\endcsname\endcsname}%
    \@xp\@tpp@temp\csname @scope@stack@def@\@scope@stack@count\endcsname%
    \@xp\let\csname @scope@stack@element@\@scope@stack@count\endcsname=\undefined%
    \@xp\let\csname @scope@stack@def@\@scope@stack@count\endcsname=\undefined%
    \edef\@scope@stack@def@count{\the\numexpr \@scope@stack@def@count - 1\relax}%
}

% Localizes a definition to the current scope
% @param #1#2       a macro definition like \def\foo
\def\localize#1#2{\@scope@stack@push{#2}#1#2}

\def\beginscope{%
    \@scope@stack@push\@scope@stack@def@count%
    \def\@scope@stack@def@count{1}%
}

\def\endscope{%
    \message{Popping \@scope@stack@def@count\ elements}%
    \repeatit{\@scope@stack@def@count}{\@scope@stack@pop}%
}

\long\def\function#1#2#3{%
    \p@array@init{\m@strip#1@param@list}{#2}%
    \def#1##1{\beginscope%
        \p@array@init{\m@strip#1@param@call@list}{##1}%
        \count1=0 %
        \def\@iterator@temp####1{%
            \p@array@index{\m@strip#1@param@call@list}{\count1}%
            \advance\count1 by 1 %
            \localize\let####1\tpp@return%
        }%
        \let\p@exe=\@iterator@temp%
        \csname p@array@\m@strip#1@param@list\endcsname%
        #3%
    \endscope}%
}

